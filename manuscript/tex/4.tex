% !TeX root = ../thesis.tex
\chapter{A parametric prior for $h(t)$\label{chapter:4}}

\begin{chaptersections}{%
In this Chapter, we derive a prior for vocal tract resonance (VTR) frequencies and bandwidths.
}

\section{Introduction}

Source-filter theory asserts that the physical resonances and antiresonances of the vocal tract show up as local maxima and minima in the envelope of the power spectrum $|\htilde(x)|^2$ of the vocal tract transfer function.
These extrema are known as \emph{formants} and \emph{antiformants}, respectively, and their amplitudes, frequencies and bandwidths give a compact description of the resonant and damping (`antiresonant') behavior of the vocal tract.

% This paragraph is used in Ch1
These (anti)resonant characteristics are described naturally in the frequency domain by the transfer function $\htilde(x)$, obtained from the impulse response $h(t)$ by the inverse FT \eqref{eq:ht}.
We make the standard LTI assumption in Chapter~\ref{chapter:4} that $\htilde(x)$ is a \emph{rational transfer function} whose \emph{poles} and \emph{zeros} are identified with the aforementioned resonances and antiresonances, respectively.\footnote{%
This is equivalent to assuming that the underlying dynamics describing $s(t)$ are (have been linearised to) linear differential equations \citep{Antsaklis2006}.
It is also equivalent to assuming that the complex exponentials are a good basis for the functions in \eqref{eq:lti-full} (which is presumably true, since Fourier analysis has shown to be useful for speech analysis for more than a century \citep{Hermann1889}).
Technically, LTI systems in applications need not have rational transfer functions, but in practice the vast majority do.
That being said, we will take the view in this thesis that $\htilde(x)$ is just a convenient expansion of the true transfer function; its poles and zeros need not necessarily correspond to real resonances and antiresonances of the vocal tract, and even when this is not the case, the expansion will still be useful when it results in a realistic spectral envelope.
}
\begin{equation}
	\htilde(x;\bp,\bz) = \frac{fixme,\prod(x - z_k)}{(x - i2\pi x_k)x - i2\pi x_k}
\end{equation}

Zeros and poles come in conjugate pairs (or are real if not in pairs) because the coefficients of the numerator and denominator polynomials in $\htilde(x)$ are real, and roots of such polynomials come in conjugate pairs or are real.
These coefficients are real because $\htilde(x)$ is the FT of the real-valued function $h(t)$.
Another way of seeing this is that the LTI system describes linear differential equations with real coefficients, which after Fourier transformation relate directly to the coefficients of the numerator and denominator polynomials.

A rational transfer function $\htilde(x)$ in the frequency domain implies that the impulse response $h(t)$ in the time domain consists of a real-valued superposition of $K$ exponentially decaying sinusoids, one for each complex pole in $\bp = p_{1:K}$, with complex amplitudes $\bc = c_{1:K}$:
\begin{equation}
	h(t; \bc, \bp) = (t \geq 0) \times \sum_{k=1}^K (c_k e^{p_k t} + c_k^* e^{p_k^* t}). \label{eq:hsum}
\end{equation}
Each pole $p_k = -\pi y_k + i 2\pi x_k$ determines the bandwidth $y_k > 0$ and frequency $x_k > 0$ of the associated \emph{vocal tract resonance} (VTR).
The zeros of $\htilde(x)$ are implicit in \eqref{eq:hsum}: their bandwidths and frequencies are determined by the amplitudes $\bc$ and the poles $\bp$ and we usually leave it this way as they are generally of minor importance.

The importance of elementary decaying sinusoids in the time domain of the form \eqref{eq:hsum} is emphasized by \cite{Chen2019} who write that ``a superposition of [these] elementary waves constitutes voice.'' % This phrase is not used in Ch1
This real-valued superposition is more convenient mathematically for our purposes than a complex rational function, so we express our priors in the time domain as is quite common in Bayesian spectrum analysis \citep{Turner2014}.

Chapter~\ref{chapter:4} introduces two priors for the impulse response of the vocal tract: the \emph{pole-zero} (PZ) prior
\begin{equation}
	\hyperref[chapter:4]{\boxed{\textbf{Parametric prior:} \quad h(t) \sim \pi_\text{PZ}(h(t))}}
\end{equation}
and the \emph{all-pole} (AP) prior
\begin{equation}
	\hyperref[chapter:4]{\boxed{\textbf{Parametric prior:} \quad h(t) \sim \pi_\text{AP}(h(t))}}
\end{equation}
Samples from these priors in the frequency domain are shown in Figure~\ref{fig:hsamples}.

As the name suggests, the difference between the AP and PZ priors is that the AP prior lacks zeros: it therefore cannot easily model antiresonances of the vocal tract, which for example occur with nasal consonants [\underline{n}ow si\underline{ng}].
This is often used as argument against the use of all-pole transfer functions in speech processing \citep[for example][]{Mehta2012}.

---

GIF methods use almost always AP transfer functions:

``Speech carries information about phonemes, the basic units of spoken language. Among phonemes, the current study, like almost all previous GIF investigations, focuses on non-nasalized vowels. This category of speech sounds has been prevalent in GIF studies for two reasons: non-nasalized vowels are always voiced (i.e. generated by the vibration of the vocal folds) and their vocal tract lacks coupling to the nasal tract, thereby justifying the use of all-pole type of models [27] for the vocal tract.'' \citep{Bleyer2017}

"There certainly will be
antiresonances in any vocal tract shape that contains the equivalent of
a side tube, such as the oral cavity in the case of a nasal sound. LPC
analysis is not reliable for nasalized vowels" \citep[][p.~124]{Maurer2016}

The GIF methods differ mainly in how the vocal tract transfer function $\htilde(x)$ is estimated, but most are based on LP analysis, which assumes that the
vocal tract transfer function can be approximated by an all-
pole filter \cite{Kadiri2021}.

---

Need zeros: ``None of the methods discussed so far account for possi-
ble zeroes in the spectrum. This is a serious deficit given
that zeroes are components of nasality (Fujimura, 1962) and
side channels (Stevens, 1998, p. 549), and they occur during
the open phases of the glottis (Ananthapadmanabha and
Fant, 1982; Plumpe et al., 1999). LPC analysis on speech
containing zeroes can lead to serious errors in locating poles
(e.g., Gutowski et al., 1978). Atal and Hanauer (1971, p.
638) discuss the issue at length, acknowledging that the all-
pole model intrinsic to LPC cannot model zeroes, and anti-
resonances do occur in many speech sounds. However, they
argue that the location of a pole is considerably more
important perceptually than the location of a zero.'' \citep{Whalen2022}

And antiformants may be much more important than anticipated:

The implications of the male emphasis may have reached even
to theory; Titze (1989, p. 1699) commented, ‘One wonders, for exam-
ple, if the source-filter theory of speech production would have taken
the same course of development if female voices had been the primary
model early on.’ Klatt and Klatt (1990, p. 820) remarked on the same
point: ‘informal observations hint at the possibility that vowel spectra
obtained from women’s voices do not conform as well to an all-pole
[i.e. all formant] model, due perhaps to tracheal coupling and source/
tract interactions.’ The acoustic theory for vowels […] assumed that the
vocal tract transfer function is satisfactorily represented by formants
(poles) and that antiformants (zeros) are required only for modifications
such as nasalization. It is advisable to bear in mind that this theory is
predicated largely on the characteristics of adult male speech and that
it may have to be altered to account for the characteristics of both
children and women.” (Kent \& Read, 2002, pp. 189–190) \citep[][p.~126]{Maurer2016}


But this argument can be misleading, since zeros in $\htilde(x)$ need \emph{not} imply strong antiresonances: if their bandwidths are sufficiently large, they simply contribute to the overall shape of the spectrum $\htilde(x)$ rather than pinpointing actual physical antiresonances due to nasal coupling or the location of additional sources along the vocal tract \citep{Peterson1966}.
This problem is not specific to the zeros: poles with large bandwidths can and routinely do shape the spectrum in exactly the same way \citep[][p.~177]{Fulop2011}.

It is for this reason that we prefer to think of \eqref{eq:hsum} as a $K$th order (partial fraction) \emph{expansion} of the true vocal tract transfer function, where $(\bc,\bp)$ are simply free parameters to be determined from the data, rather than a physical model of the resonance characteristics of the vocal tract where each pole identifies a physical resonance and each zero identifies a physical antiresonance.
Of course, we \emph{hope} for the latter, and in fact we can use Bayesian model selection to test the plausibility of different expansions for the same speech data.

The same issue underlies a conceptual problem with the concept of \emph{formants} \citep{Whalen2022}.
Poles are routinely identified with formants but they can be shifted.
Multiplets can and do occur in many spectral measurements, including those of speech \citep{VanSoom2021}.
Note that \emph{formants} for us formally are basically statistics derived from these spectra as they are "general broad peaks" that need not have the typical form.
Show how formants are derived from VTRs.
In practice though they sometimes be identified as VTRs if $K$ is small number.
But in principle VTRs are for us nothing more than expansion coefficients in an expansion of the rational transfer function.
(too much detail).
A formant is essentially a proxy for a VTR because it is a local maximum in the spectral envelope, which includes source effects.
For example, harmonic "attraction" of formants due to the $F_0$: clearly a source effect. \citep{Whalen2022}

Another issue with the AP prior for $K$ poles is that it has an expected characteristic $12K$ dB/octave spectral tilt, which compares unfavorably to the expected value in real speech (around -2 dB/octave).
By contrast, the expected spectral tilt of the PZ prior is independent of $K$ and simply -6 dB/octave.

Nevertheless, AP transfer functions are generally thought to be sufficiently expressive to represent the vocal tract filter  \citep{Flanagan1965,Fulop2011} and have been computationally the preferred choice since the very beginning of digital acoustic phonetics \citep{Atal1971}.
This is because they admit an efficient representation known as \emph{linear prediction} \citep{Markel1976}, which is the traditional workhorse used in acoustic analysis.

The advantage of the AP prior lies in its computational efficiency: it has less free parameters (no zeros are modeled, so the amplitudes of the decaying sinusoids are determined entirely by the poles) and admits 


\begin{equation}
h(t; \bg, \bx, \by) = (t \geq 0) \times \sum_{k=1}^K (g_k \cos(2\pi i x_k) + g_{k+K} \sin(2\pi i x_k)) \exp{-\pi y_k t} \label{eq:hparam}
\end{equation}

We assume a causal rational transfer function which has a real-valued causal impulse response of the form
\begin{equation}
    h(t; p, c) = (t \geq 0) \times \sum_{k=1}^K c_k e^{p_k t} + c_k^* e^{p_k^* t}
\end{equation}
where the $K$ poles and complex amplitudes $p_k, c_k \in \mathbb{C}$ determine $2K$ real parameters:
\begin{align}
    p_k &= -\pi y_k + 2\pi i x_k \\
    c_k& = \frac{a_k - ib_k}{2}
\end{align}
The poles $p_k$ describe the \emph{vocal tract resonances} (VTRs) in terms of frequency $x_k$ and bandwidth $y_k$.
The complex amplitudes $c_k$ describe the amplitude $A_k = \sqrt{a_k^2 + b_k^2}$ and phase $\varphi_k = \arctan(b_k/a_k)$ of these resonances.

In practice formants are generally considered to be more important than antiformants as antiformants are less important perceptually due to auditory masking by the surrounding formants \citep{Schroeder1999}.
In addition, computational formant tracking admits an efficient representation known as \emph{linear prediction} \citep[LP;][]{Markel1976}, which is the traditional workhorse used in acoustic analysis, while ``antiformant tracking remains a challenging task'' \citep[][p.~11]{Mehta2012}.
For these reasons we mostly focus on formants in this thesis.

\section{Harmonic coupling}
% In Ch1 we only introduce harmonic attraction, which is called harmonic coupling of the second kind here

% Here follows an extended version of what we wrote in Section Contributions and merits of BNGIF in Chapter 1

Unlike other primates humans have the ability to control pitch and timbre relatively independently.
Permitting pitch and timbre to be mapped to source and filter respectively, independence of pitch and timbre translates to independence of source and filter, one of the central simplifying assumptions made in source-filter theory.
Source-filter interactions do exist, of course, and there is a somewhat elusive subclass of these that engage in what we call \emph{harmonic coupling}, which has important consequences for formant tracking and GIF algorithms.
We define harmonic coupling in this thesis as a source-filter interaction specifically between the fundamental frequency $F_0$ and the formant frequencies $\bF = (F_1, F_2, \ldots)$, resulting in observable correlations between these two statistics, thus breaking source-filter independence.\footnote{%
	An example of a non-harmonic coupling type of source-filter interaction is the coupling of the vocal tract to the subglottal cavities during the open phase of the glottis, which results in a time-dependent modulation of the formant bandwidths $\bB$ \citep{Pinson1963}.
	This also serves as an example of an important source-filter interaction which BNGIF ignores.
}

The first kind of harmonic coupling between $F_0$ and $\bF$ is mediated through the vocal tract length $L$, measured from the glottis to the lips, and primarily reflects the speaker sex: women tend to have both higher fundamental frequencies and higher formant frequencies than men \citep{Goldstein1980,Fant1966}.
Since $F_0$ correlates inversely proportionally with $L$ (bigger heads correlate with lower voices) and $\bF$ correlates inversely proportionally with $L$ (uniform scaling hypothesis \citep{Fant1966}), $F_0$ correlates proportionally to $\bF$.
Managing this systematic correlation for speech processing applications is known as vocal tract length normalization \citep{Cohen1995}.

The second kind of harmonic coupling between $F_0$ and $\bF$ does not have physiological substance; it is rather an inconvenient side effect of applying the spectral smoothing inherent in LP analysis to voiced speech, such that the analysis window contains several pitch periods, which is typically the case in conventional acoustic phonetics.
High $F_0$ values pose challenges in accurately measuring formants %\citep{Kent2018}
due to a phenomenon \cite{Whalen2022} call ``harmonic attraction.''
This refers to the fact that, as a result of spectral smoothing, the peaks of the estimated envelope obtained through LP are highly biased toward the harmonic partials of a speech spectrum \citep{Yoshii2013}, such that the estimated formant frequencies $\bF$ tend to align with multiples of $F_0$, thus causing systematic correlations between $F_0$ and $\bF$.
In the case that $F_0 \approx F_1$, known as $F_0-F_1$ congruence \citep{Kent2018}, harmonic attraction can be especially challenging, since LP analysis, as a spectral smoothing device, lacks a conceptual model for separating the two -- in contrast to, for example, the auditory processing regions of our brains \citep{Schnupp2011}.

"In the case of female speech, formant analysis is extremely dif-
ficult. The fundamental frequency is so high that formants are often
poorly defined. […] We had difficulties in determining the position of a
formant in about 40% of the 300 vowel segments, if no a priori knowl-
edge was used.” (Van Nierop et al., 1973" \citep[][p.~124]{Maurer2016}
}

These issues have implications for GIF as errors in the spectral envelope affect the reconstruction of $u(t)$, in addition to the impact of shorter closing times on sampling.
The GIF literature acknowledges this problem, as inverse filtering methods tend to handle male vowels well but face difficulties with higher $F_0$ values, limiting their applicability to women and children.
Another manifestation of harmonic attraction is the $F_0-F_1$ congruence, which occurs when the values of $F_0$ and $F_1$ are close, such as in the case of close rounded vowels, which remain ``a major factor that limits the applicability of inverse filtering algorithms to accurate glottal flow estimation from continuous speech.'' \citep[][p.~17]{Chien2017}.

BNGIF has the advantage of being a joint-source filter optimization method here and avoids harmonic attraction completely; indeed, it can even model harmonic coupling (first category) and is therefore on the lookout for harmonic attraction effects which are challenging for inverse filtering methods.
The prior for $h(t)$ can incorporate basic $F_0$ information that encodes gender to some extent.
In addition it is a Bayesian method, and these are known for their superior performance in high $F_0$ environments.


\section{Discussion}

\citep{Auvinen2014} also use a prior that imposes a hard constraint on ordered $\bx$ (p. 1145), whereas we only impose this softly.

\citedepigraph{%
For the speech scientist the vocal tract, originally designed by evolution for breathing, eating, and drinking before it was coopted for speaking, is an acoustic resonator with variable geometry under control of the (sober) speaker.
}{Schroeder (1999)}{Schroeder1999}

Antiresonances may result from a number of different factors in an acoustic
tube, such as branch connections or the location of sources along the acoustic network. \citep{Peterson1966}

Adopting the time domain for expressing PZ and AP is counterintuitive in the literature, but has yielded some very nice advantages:
\begin{enumerate}
	\item Realization that AP needs an overall gain factor!
	\item Prior for amplitudes based on power constraints
	\item Zeros are implicit in the amplitudes and poles; so their number is also implicit and does not need to be determined beforehand. This is traditionally a more difficult task prone to manual guessing and tuning. For example: \citep[][pp.~194-197]{Fulop2011} and \citep[][p.~11]{Mehta2012}: ``Nevertheless, the proposed approach allows
	the user the option of tracking antiformants during select
	speech regions of interest. Potential improvements here
	include the use of formal statistics tests for detecting the
	presence of zeros within a frame prior to tracking them''. There is no need for such tests here.
	\item Switching to amplitude notation instead of amplitude and phase implicit in poles and zero approach brings analytical advantages just like \citep{Bretthorst1988} over \citep{Jaynes1987}.
	\item This follows the symmetry between $h(t)$ and $u'(t)$: either one can be expanded into amplitudes who can be integrated out.
\end{enumerate}

\end{chaptersections}

\begin{chapterappendices}{6}
	
\section{The definition of a formant}

The way formants are defined and operationalized in this thesis does not necessarily correspond to other uses of the term `formant' in the literature. (And likewise for `antiformant.')
Acoustic phonetics lacks unanimous consensus on this matter:
\begin{quote}
	Unfortunately, the common definition between a formant and a [vocal tract] resonance is yet to be established. [...] Authors who are heavily invested in formant frequency analysis are encouraged to be as clear as possible about the relation between a peak in the output spectrum and a presumed resonance of the
	vocal tract. \citep[][p.~3006]{Titze2015}
\end{quote}
Our usage of formants is synonymous with ``resonances of the
vocal tract'' and do \emph{not} ``refer to peaks in the output spectrum,'' unless explicitly stated otherwise.
This convention is proclaimed on p.~\pageref{chapter:conventions}.

The term `formant' was originally defined in the 19th century by \cite{Hermann1889} as a local maximum in the envelope of the \emph{speech} power spectrum; that is, a property of the output $|\stilde(x)|^2$ rather than the filter $|\htilde(x)|^2$.
This usage is common in the literature \citep{Titze2015}, but we prefer the filter variant as acoustic phoneticians are generally concerned with the `intrinsic' filter characteristics of the vocal tract (which have to be inferred) rather than the `extrinsic' system output (which is readily measurable).

Indeed, formants are almost always measured using linear prediction (LP) methods whose order needs to be tuned ``in order to represent
the poles \emph{of the vocal tract transfer function} adequately'' \citep[emphasis ours;][p.~639]{Atal1971}.
The fact that the LP poles (intend to) physically measure `intrinsic' vocal tract resonances was also emphasized more recently by \cite[][p.~179]{Fulop2011}.

\section{Nested sampling of a posterior with exchange symmetry\label{nested-exchange}}

\shloppy{Rewrite this section into something short}

While nested sampling of course has no theoretical problem with calculating $Z$ for a posterior exhibiting exchange symmetry, the latter does pose some difficult practical issues, which $\pi_3$ tries to address in a plug and play approach; i.e., compatible with off-the-shelve exploration algorithms which can be found in libraries such as \dynesty.

The main practical issue is that for most exploration algorithms, the posterior modes due to exchange symmetry may (will) prove to be too \emph{numerous} and too \emph{sticky} to enable convergence of the nested sampling run.

\paragraph{Numerous} because many exploration algorithms work with topology and retain some kind of bounds to restrict the prior subject to the current lowest likelihood value, such as elliptical bounds based on clustering the particles by their mean and covariances. If the live particles are spread along many modes, this will cause problems to estimate accurate bounds and will prevent convergence, as practical experience bears out.

\paragraph{Sticky} because the `gates' \citep[Sec.~9.5]{Sivia2006} between primary and induced modes must be explored
in order to arrive at the correct value of $Z$, and practical exploration algorithms might be too biased towards the sticky modes such that they miss these important gates. This is a subtle point but important to understand how nested sampling infers the value of $Z$ for posterior landscapes with exchange symmetry. The gates are crucial because they create regions of enclosed prior mass $X$ in which likelihood values are larger than they would be if the model would not have the exchange symmetry, because the latter interpolates smoothly between primary and induced modes. Such early discovery (while $X$ is still closer to 1) of larger likelihood values timely corrects the value of $Z$ upwards to account for the extra mass contained by the induced modes. Indeed, once the current likelihood lower bound $\likelihood^*$ is such that all enclosed regions of the prior for which $\likelihood(\bx) > \likelihood^*$ have disconnected into islands, and there is at least one particle in each primary mode or one of its induced copies, nested sampling does not make any distinction between the problem with or without exchange symmetry (assuming adequate exploration and no mode die-off). The upward revision of $Z$ must happen entirely before $\likelihood$ reaches that $\likelihood^*$, and it happens by the aforementioned early discovery of higher likelihood values residing in those gates. If $K$ is moderately large, the number of possible gates grows quickly, as can be seen in the pairwise projections in Figure~\ref{label-switching}. In addition, multiplets (for example modes close to the bissectrice in Figure~\ref{label-switching}) create ample opportunity for such gates. A proliferation of gates nearby sticky modes poses practical problems for exploration.

\section{Interpretation of $\pi_3$ in acoustic phonetics\label{pi-theory}}

\shloppy{Rewrite this section into something short}

It is easily verified that all three priors $\pi_i$ are scale-free distributions, i.e., $\pi_i(c\bx) \propto \pi_i(\bx)$ with $c > 0$ \citep{Newman2005}. 
Assuming the uniform scaling hypothesis \citep[e.g.,][]{Fant1966}, the scale transformation $\bx \rightarrow c\bx$ corresponds to a uniform rescaling of the vocal tract such that its length $L \rightarrow L/c$.\footnote{%
For example, \citep{Goldstein1980} has estimated that $L$ for females is about 20\% shorter than $L$ for males, and indeed one finds that on average female formants are about 20\% higher than male ones \citep{Fant1966}.
}
The $\pi_i$ thus succeed in representing information about the resonance frequencies $\bx$ in a way that is \emph{independent of the speaker's vocal tract length}, which is the major source of inter-speaker variability after vowel type \citep{Turner2009}.\footnote{%
The scale-free criterion $p(cx|\beta) \propto p(x|\beta)$ is not to be confused with requiring invariance of functional form under a given transformation $\{x,\beta\} \rightarrow \{x',\beta'\}$ as in Jaynes' transformation invariance principle (see Appendix~\ref{invariance}); indeed, the former is much more stringent than the latter.
In Jaynes’ method, invariance of functional form is required under transformations \emph{between problems}, such that the sample and parameter space $\{x, \beta\}$ are transformed simultaneously \citep[App.~A]{Jaynes1989a}.
In contrast, the scale-free criterion only involves a transformation of the sample space $x$ (i.e., $x \rightarrow cx$), without a possible `countertransformation' of the parameter space $\beta \rightarrow \beta'$ to `compensate' for the transformation of the sample space.
For example, any distribution of the form $p(x|\beta) = (1/\beta) h(x/\beta)$ (e.g., a zero-mean Gaussian) is invariant in form under $\{x, \beta\} \rightarrow \{cx, c\beta\}$, while not necessarily scale-free; in fact, the only one-dimensional scale-free distribution is the Pareto distribution \eqref{pareto} \citep{Newman2005}.
}

This is true even for $\pi_3$, despite the increased amount of prior information it would typically represent (Table~\ref{infotable}); in general, in the maximum entropy framework, the symmetries of the invariant measure are not preserved under adding constraints as in \eqref{DKL}.
That this is not the case for the invariant measure $m(\bw)$ is due to the fact that the scale invariance is built into the $\bx \rightarrow \bw$ transformation \eqref{tf}.

Thanks to this built-in quality, the $\bw$ space is also a ``natural space'' to describe vowel type information \citep{Turner2003}.
This fact can be exploited, for example, when constructing a new prior $p(\bx|x_0,X)$ (where $X$ is a dataset of previously observed $\bx$ samples such as \citep{Peterson1952}) which is to be informative (for example to represent prior knowledge that the data will be a open vowel) but still independent of the speaker's vocal tract length.
This can be done by transforming $X$ and processing that information in that space through, say, mixture modeling, or maximum entropy density estimation based on empirical moments $\overline{w^k}$ \citep{Bretthorst2013}, and then transforming the obtained density back to obtain the desired $p(\bx|x_0,X)$. 

Furthermore, we note that the log ratio transformation $w_k = \log(x_k/x_{k-1})$ in \eqref{tf} exhibits several useful properties which have been disparately observed in the literature of acoustic phonetics.
For example, ratios of consecutive frequencies $(x_k/x_{k-1})$ are the foundation of formant ratio theory \citep{Lloyd1890}.
The \emph{log} of these ratios, i.e., $w_k$, is the preferred representation in Miller's classical theory of vowel perception \citep{Miller1987}.
The empirical first moments $\bbarw$ used in \eqref{DKL} also play a role in vowel normalization methods \citep{Johnson2018}, and we note in passing that they avoid the amplification of the error in the frequency in the denominator which is ``likely to have hampered efforts to normalize for acoustic scale using formant ratios'' \citep[p.~2384]{Turner2009}.\footnote{%
The magnitude of the error of the log ratio of two numbers $a$ and $b$ does not depend on whether we divide $b$ by $a$ or $a$ by $b$.
This is not true for the error of the ratio alone, since $\var{\log{b/a}} = \var{b}/b - \var{a}/a$ while $\var(b/a) = (b/a) \var{\log{b/a}}$.
}

While these connections are of course specific to the domain of acoustic phonetics, we might expect similar advantageous connections in other fields where resonance frequencies play important roles.

\subsection{Another way of looking at it}

The maximum entropy framework is invariant under transformations, but this does not remove the arbitrariness in choosing which moments to fix.
This is similar to the fact that specifying a flat prior in one coordinate frame is not flat in another: we need to find the appropriate coordinate frame, and this choice is `arbitrary'; i.e., it is not prescribed by probability theory, because it is one of the ways information is encoded into the algebra.

From this point of view, the previous paragraphs of this Appendix not so much interpret $\pi_3$ as answer the question, ``why fix the particular moments $\expval{w_k} = \expval{\log(x_k/x_{k−1})}$ and not any other function of $x_k$?''
The answer, in short, is that the theoretical properties of the function $\log(x_k/x_{k−1})$ make its expectation value a meaningful quantity to fix, at least within the domain of acoustic phonetics.
The fact that the $\overline{w_k} = 1/\lambda_k$ are expressible in terms of something much more likely to be known, i.e., in terms of $\bbarxnull$, is merely an additional convenience.

\end{chapterappendices}
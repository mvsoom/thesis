\citep{Chien2017}
\citep{Alku2011} p. 600

There are several problems with the current state of the art of inverse filtering in acoustic phonetics. One of the main issues is the difficulty in accurately estimating the filter in order to separate the glottal source from the vocal tract filter. This is due to the fact that the vocal tract filter is not known a priori and can vary widely across speakers and utterances. Another problem is the lack of information about the glottal source signal itself, which is traditionally viewed as 'the signal'.

Parametric models, such as the Liljencrants-Fant (LF) model or the Rosenberg-Klatt model, have been used in several source-filter separation methods, but they have limitations in capturing the real glottal flow. In addition, existing inverse filtering methods and parametric models produce outputs that are often too smooth, which is unrealistic.

Recent pitch-synchronous approaches for linear predictive coding (LPC) have been designed to overcome F0 bias but require very accurate glottal pulse information, usually provided by an electroglottography (EGG) signal. However, requiring EGG to be recorded along with the audio signal severely restricts the usefulness of pitch-synchronous LPC.

Another issue is the sensitivity of many source-filter separation methods to glottal closure instant (GCI) detection. Errors in GCI estimation can result in severe distortion of the estimated glottal excitation. However, GCI detection is necessary for recent pitch-synchronous LPC approaches and other source-filter separation methods.